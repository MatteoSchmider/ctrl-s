\section{Einleitung}

% Anfang: Abschaltung letztes AKW in DE
% Zur Abschaltung der Kohle/Gas-KWs in DE muss:
% 1.) Erneuerbare Energie in ausreichender Menge zur Verfügung stehen
%   => Die BEG verdient hieran schon mit
% 2.) Erneuerbare Energie gespeichert werden, um 24/7 die Energieversorgung zu gewährleisten
%   => Hieran verdient die BEG noch nicht, allerdings verdient hieran generell kaum jemand
%   => da niemand es tut muss es unwirtschaftlich sein?!
% Bestätigung der Behauptung, es sei unwirtschaftlich: 
% \href{https://www.bundesnetzagentur.de/SharedDocs/Downloads/DE/Sachgebiete/Energie/Unternehmen_Institutionen/ErneuerbareEnergien/Speicherpapier.pdf?__blob=publicationFile&v=5}{Bericht der Bundesnetzagentur} 
% Zitat: "Ein rentabler Betrieb von Stromspeichern wäre im gegenwärtigen Marktumfeld sehr schwierig, wenn die
% Stromspeicher für den Strom, den sie bei der Einspeicherung verbrauchen, alle üblichen Stromnebenkosten
% und die Strombeschaffungskosten zu tragen haben."
% Das Urteil der BNetzA weißt in ihrem Urteil darauf hin, dass in der derzeitigen Marktlage Speicher technisch und ökonomisch nicht notwendig sind,
% Dies aber bei steigenden EE-Anteilen am Gesamtstrommarkt bald werden.
% Insbesondere ist das Fehlen geeigneter Speicher für EE-Einspeiser schlecht zu beurteilen:
% Damit Speicher ökonomisch werden muss der Strompreis bei Hochproduktionszeiten stark sinken und/oder bei Niedrigsproduktionszeiten stark steigen
% => Dass heißt geringere Stromerträge für die PV-Anlagen der BEG
% => auch Windkraftanlagen, welche einen möglichen zukünftigen Wachstumspfad der BEG darstellen sind schon heute betroffen (siehe SMARD Strompreis-Abhängigkeit von Wind in Herbst/Winter)
% Bei genauerer Betrachtung des BNetzA-Zitats fällt jedoch auf: Bei Einsparen der Stromnebenkosten und Strombeschaffungskosten ist ein Betrieb vielleicht doch möglich
