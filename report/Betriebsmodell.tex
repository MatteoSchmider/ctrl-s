\section{Betriebsmodell}

% Der Sinn von Stromspeichern ist ja gerade übercshüssige Erneuerbare Energie aus dem Netz zu entnehmen
% Vereinfacht ausgedrückt also: EE-KW -----> Netz -(1)-> Speicher -(2)-> Netz
% Argumentation der BNetzA: Es entstehen Kosten bei (1), die der Umsatz in (2) ausgleichen muss, die neben dem Strompreis auch Stromnebenkosten enthalten
% Diese Kosten machen in Deutschland bekanntermaßen etwa 50% der Stromkosten aus und sind nach Bewertung der BNetzA in selbigem Bericht auch unumgänglich,
% da andere Stromerzeuger ebenso Nebenkosten und Abgaben zur Stromgestehung erbringen. (z.B. Transport von Kohle oder Aufbereitung von Biomasse)
% Als Betreiber ausreichend großer EE-KWs ermöglicht sich jedoch eine direkte Betriebsform:
% EE-KW -(1)-> Speicher -(2)-> Netz
% Nach diesem Modell entfallen die Stromnebenkosten fast vollständig, was auch technisch begründbar ist: 
% auf dem Übertragungswerk zu einem lokalen Speicher gibt es nunmal keine ander Partei, die mitverdienen könnte.
% Eine interessante Fragestelluing liegt nun in der Auswahl der Speichertechnologie:
% Insbesondere wird hier bei der durchschnittlich sehr kleinen Größe der PV-Anlagen der BEG die Skalierbarkeit auf kleine Anlagen von Vorteil.
% Zusätzlich müssen bei der Bewertung unbedingt die Kosten auf die gesamte Lebenszeit des Speichers betrachtet werden, da sich die unterschiedlichen Typen in ihrer Lebensdauer stark unterscheiden.
% Zur groben Einordnung
